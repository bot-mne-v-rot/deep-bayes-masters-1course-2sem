\documentclass[a4paper]{article}

%plastikovye pakety

\usepackage[12pt]{extsizes}
\usepackage[utf8]{inputenc}
\usepackage[unicode, pdftex]{hyperref}
\usepackage{cmap}
\usepackage{textcomp}
\usepackage{mathtext}
\usepackage{multicol}
\setlength{\columnsep}{1cm}
\usepackage[T2A]{fontenc}
%\usepackage[english,russian]{babel}
\usepackage[english]{babel}
\usepackage{amsmath,amsfonts,amssymb,amsthm,mathtools}
\usepackage{icomma}
\usepackage{euscript}
\usepackage{mathrsfs}
\usepackage[dvipsnames]{xcolor}
\usepackage[left=2cm,right=2cm,
    top=2cm,bottom=2cm,bindingoffset=0cm]{geometry}
\usepackage[normalem]{ulem}
\usepackage{graphicx}
\usepackage{makeidx}
\usepackage{extarrows}
\usepackage[shortlabels]{enumitem}
\usepackage{listings}
\makeindex
\graphicspath{{pictures/}}
\DeclareGraphicsExtensions{.pdf,.png,.jpg}
%\usepackage[usenames]{color}
\hypersetup{
     colorlinks=true,
     linkcolor=coralpink,
     filecolor=coralpink,
     citecolor=black,      
     urlcolor=coralpink,
     }
%\usepackage{fancyhdr}
%\pagestyle{fancy} 
%    \fancyhead{} 
%    \fancyhead[RO]{
%        \includegraphics[width = 0.\textwidth]{shilat.jpeg}
%} 
%\fancyhead[CO]{ }
%\fancyhead[LO]{keba4ok} 
\newtheoremstyle{indented}{0 pt}{0 pt}{\itshape}{}{\bfseries}{. }{0 em}{ }
\renewcommand\thesection{}
\renewcommand\thesubsection{}
\renewcommand\thesubsubsection{}
%\geometry{verbose,a4paper,tmargin=2cm,bmargin=2cm,lmargin=2.5cm,rmargin=1.5cm}

%\renewcommand{\boxed}[1]{\text{\fboxsep=.2em\fbox{\m@th$\displaystyle#1$}}}

%declarations
%arrows_shorten
\DeclareMathOperator{\la}{\leftarrow}
\DeclareMathOperator{\ra}{\rightarrow}
\DeclareMathOperator{\lra}{\leftrightarrow}
\DeclareMathOperator{\llra}{\longleftrightarrow}
\DeclareMathOperator{\La}{\Leftarrow}
\DeclareMathOperator{\Ra}{\Rightarrow}
\DeclareMathOperator{\Lra}{\Leftrightarrow}
\DeclareMathOperator{\Llra}{\Longleftrightarrow}
\DeclareMathOperator{\bs}{\backslash}
%letters_different
\DeclareMathOperator{\CC}{\mathbb{C}}
\DeclareMathOperator{\II}{\mathbb{I}}
\DeclareMathOperator{\MS}{\mathbb{S}}
\DeclareMathOperator{\DD}{\mathbb{D}}
\DeclareMathOperator{\ZZ}{\mathbb{Z}}
\DeclareMathOperator{\QQ}{\mathbb{Q}}
\DeclareMathOperator{\RR}{\mathbb{R}}
\DeclareMathOperator{\NN}{\mathbb{N}}
\DeclareMathOperator{\HH}{\mathbb{H}}
\DeclareMathOperator{\PP}{\mathbb{P}}
\DeclareMathOperator{\EE}{\mathbb{E}}
\DeclareMathOperator{\LL}{\mathscr{L}}
\DeclareMathOperator{\KK}{\mathscr{K}}
\DeclareMathOperator{\GA}{\mathfrak{A}}
\DeclareMathOperator{\MA}{\mathbb{A}}
\DeclareMathOperator{\GB}{\mathfrak{B}}
\DeclareMathOperator{\MB}{\mathbb{B}}
\DeclareMathOperator{\GC}{\mathfrak{C}}
\DeclareMathOperator{\GD}{\mathfrak{D}}
\DeclareMathOperator{\GF}{\mathfrak{F}}
\DeclareMathOperator{\GN}{\mathfrak{N}}
\DeclareMathOperator{\Rho}{\mathcal{P}}
\DeclareMathOperator{\FF}{\mathcal{F}}
\DeclareMathOperator{\MI}{\mathcal{I}}
%common_shit
\DeclareMathOperator{\Ker}{Ker}
\DeclareMathOperator{\spann}{span}
\DeclareMathOperator{\Frac}{Frac}
\DeclareMathOperator{\Imf}{Im}
\DeclareMathOperator{\cont}{cont}
\DeclareMathOperator{\id}{id}
\DeclareMathOperator{\ev}{ev}
\DeclareMathOperator{\lcm}{lcm}
\DeclareMathOperator{\chard}{char}
\DeclareMathOperator{\codim}{codim}
\DeclareMathOperator{\rank}{rank}
\DeclareMathOperator{\ord}{ord}
\DeclareMathOperator{\End}{End}
\DeclareMathOperator{\Ann}{Ann}
\DeclareMathOperator{\Real}{Re}
\DeclareMathOperator{\Res}{Res}
\DeclareMathOperator{\Rad}{Rad}
\DeclareMathOperator{\disc}{disc}
\DeclareMathOperator{\rk}{rk}
\DeclareMathOperator{\const}{const}
\DeclareMathOperator{\grad}{grad}
\DeclareMathOperator{\Aff}{Aff}
\DeclareMathOperator{\Lin}{Lin}
\DeclareMathOperator{\Prf}{Pr}
\DeclareMathOperator{\Iso}{Iso}
\DeclareMathOperator{\cov}{cov}
\DeclareMathOperator{\argmax}{argmax}
\DeclareMathOperator{\argmin}{argmin}
\DeclareMathOperator{\tr}{\textbf{tr}}
%specific_shit
\DeclareMathOperator{\Tors}{Tors}
\DeclareMathOperator{\form}{Form}
\DeclareMathOperator{\Pred}{Pred}
\DeclareMathOperator{\Func}{Func}
\DeclareMathOperator{\Const}{Const}
\DeclareMathOperator{\arity}{arity}
\DeclareMathOperator{\Aut}{Aut}
\DeclareMathOperator{\Var}{Var}
\DeclareMathOperator{\Term}{Term}
\DeclareMathOperator{\sub}{sub}
\DeclareMathOperator{\defr}{def}
\DeclareMathOperator{\Sub}{Sub}
\DeclareMathOperator{\Atom}{Atom}
\DeclareMathOperator{\FV}{FV}
\DeclareMathOperator{\Sent}{Sent}
\DeclareMathOperator{\Th}{Th}
\DeclareMathOperator{\supp}{supp}
\DeclareMathOperator{\Eq}{Eq}
\DeclareMathOperator{\Prop}{Prop}

\DeclareMathOperator{\fquad}{\qquad \qquad}
\DeclareMathOperator{\equad}{\qquad \qquad \qquad \qquad}
%env_shortens_from_hirsh            
\newcommand{\bex}{\begin{example}\rm}
\newcommand{\eex}{\end{example}}
\newcommand{\ba}{\begin{algorithm}\rm}
\newcommand{\ea}{\end{algorithm}}
\newcommand{\bea}{\begin{eqnarray*}}
    \newcommand{\eea}{\end{eqnarray*}}

    \newcommand{\abs}[1]{\lvert#1\rvert}
    \newcommand{\bp}{\begin{prob}}
        \newcommand{\ep}{\end{prob}}
        \newcommand{\be}{\begin{ex}}
        \newcommand{\ee}{\end{ex}}
    
    

%ya_ebanutyi
\newcommand{\resetexlcounters}{%
  \setcounter{prob}{0}%
} 
\newcommand{\resetremarkcounters}{%
  \setcounter{remark}{0}%
} 
\newcommand{\reseconscounters}{%
  \setcounter{cons}{0}%
} 
\newcommand{\resetall}{%
    \resetexlcounters
    \resetremarkcounters
    \reseconscounters%
}
\newcommand{\cursed}[1]{\textit{\textcolor{coralpink}{#1}}}
\newcommand{\de}[3][2]{\index{#2}{\textbf{\textcolor{coralpink}{#3}}}}
\newcommand{\re}[3][2]{\hypertarget{#2}{\textbf{\textcolor{coralpink}{#3}}}}
\newcommand{\se}[3][2]{\index{#2}{\textit{\textcolor{coralpink}{#3}}}}

\lstset{frame=tb,
  language=Haskell,
  aboveskip=3mm,
  belowskip=3mm,
  showstringspaces=false,
  columns=flexible,
  basicstyle={\small\ttfamily},
  numbers=none,
  numberstyle=\tiny\color{gray},
  keywordstyle=\color{blue},
  commentstyle=\color{dkgreen},
  stringstyle=\color{mauve},
  breaklines=false,
  breakatwhitespace=true,
  tabsize=3
}

\definecolor{dkgreen}{rgb}{0,0.6,0}
\definecolor{gray}{rgb}{0.5,0.5,0.5}
\definecolor{mauve}{rgb}{0.58,0,0.82}

%envirnoments
\theoremstyle{indented}
\newtheorem{theorem}{Теорема}
\newtheorem{lemma}{Лемма}
\newtheorem{alg}{Алгоритм}
\newtheorem*{etheorem}{Theorem}
\newtheorem{elemma}{Lemma}
\newtheorem{ealg}{Algorythm}
\newtheorem{st}{Статья}

\theoremstyle{definition} 
\newtheorem{defn}{Определение}
\newtheorem*{exl}{Пример(ы)}
\newtheorem{prob}{ }
\newtheorem{problem}{Задача}
\newtheorem{edefn}{Definition}
\newtheorem*{eexl}{Example(s)}
\newtheorem{eproblem}{Problem}

\theoremstyle{remark} 
\newtheorem*{remark}{Примечание}
\newtheorem*{hint}{Подсказка}
\newtheorem*{cons}{Следствие}
\newtheorem{exer}{Упражнение}
\newtheorem{stat}{Утверждение}
\newtheorem*{prop}{Свойство(а)}
\newtheorem*{sol}{Решение}
\newtheorem*{ans}{Ответ}
\newtheorem*{eremark}{Remark}
\newtheorem*{ehint}{Hint}
\newtheorem*{econs}{Corollary}
\newtheorem{eexer}{Excercise}
\newtheorem{estat}{Statement}
\newtheorem*{eprop}{Property(ies)}
\newtheorem*{esol}{Solution}
\newtheorem*{eans}{Answer}

\definecolor{coralpink}{rgb}{0.8, 0.2, 0.2}


\newcommand{\reset}{%
  \setcounter{prob}{0}%
} 

