\[
    x \in \RR^n, z \in \RR^d, X = (x_1, x_2, \ldots, x_n),
\]
\[
    p(x,z \mid \theta) = p(x \mid z, \theta) p(z) = \MN (x \mid \mu + Wz, \sigma^2 I) \MN (z \mid 0, I)
.\]
Maximum likelihood
\[
    \theta_{ML} = \argmax_\theta p(X\mid \theta) = 
    \argmax_\theta \prod_{i=1}^n p(x_i \mid \theta)
.\]

An alternative is EM algorithm:
\begin{equation*}
    \begin{aligned}
        \text{E-step:} \quad &p(Z \mid X, \theta) =\prod_{i=1}^n p(z_i \mid x_i, \theta), \\ 
        \text{M-step:} \quad &\mathbb{E}_{z_i \mid x_i, \theta} \log p(X, Z \mid \theta) \to \max_{\theta}.
    \end{aligned}
\end{equation*}

This is a generative model: we sample $z \sim \MN (z \mid 0, I)$, then $\MN (x \mid \mu + Wz, \sigma^2 I)$. It's a very simple one; we generalize it by taking 
\[
    p(x,z \mid \theta) = p(x \mid z, \theta) p(z) = \MN(x \mid \mu(z, \theta), \operatorname{diag}(\sigma^2(z, \theta))) \MN (z \mid 0, I),
\]
where $\sigma$ and $\mu$ are deep neural networks. How can we then train $\sigma$ and $\mu$? Using Variation Bayesian Inference and ELBO.

\[
    \log p(X \mid \theta) = \sum_{i=1}^{n} \log p(x_i \mid \theta) \ge
    \sum_{i=1}^{n} \int q(z\mid \varphi) \log \frac{p(x_i, z_i \mid \theta)}{q(z_i \mid \varphi)} \dd z_i
\]
\[
    q(z_i \mid \varphi) =
    \argmin_{\varphi} \operatorname{KL}(q(z_i \mid \varphi) || p(z_i \mid x_i, \theta)) \approx p(z_i \mid x_i, \theta)
.\]

Given a particular sample $x_k$, we can sample $z_k$ from the posterior $q(z \mid \varphi) \approx p(z \mid x_k, \theta)$.
And then use $z_k$ to sample from $p(x\mid z_k, \theta)$.
This means we can solve the problem of generating $x$s that are similar to a given sample $x_k$ after the training of our generative model is complete (we have the approximate posterior). \\ 

$x_k$ has to come from the training data, we can't generate samples similar to a completely new $x$.
For every sample $x_i$, we have its own set of parameters $\varphi_i = (m_i, s_i)$ where $q(z_i\mid \varphi) = \MN(m_i, \operatorname{diag}(s_i^2))$. \textit{Amortized VI}:
\[
    q(z_i \mid x_i, \varphi) = N(z_i \mid m(x_i, \varphi) \operatorname{diag}(s^2(x_i, \varphi)))
.\]

\begin{figure}[htbp]
    \centering
    \begin{tikzpicture}[
        scale=2,
        node distance=2cm,
        auto,
        trapezoid/.style={
            draw,
            trapezium,
            trapezium stretches,
            shape aspect=0.8,
            minimum height=2cm,
            minimum width=4cm,
            trapezium left angle=70,
            trapezium right angle=70,
        },
        box/.style={
            draw,
            rectangle,
            minimum height=1.2cm,
            minimum width=2cm,
        },
        rect/.style={
            draw,
            rectangle,
            minimum height=4cm,
            minimum width=1cm,
        },
        arrow/.style={
            -latex,
            thick
        }
    ]

    \node[rect] (data) {Data};
    \node[trapezoid, shape border rotate=270, right of=data, node distance=3cm] (encoder) {Encoder};
    \node[box, right of=encoder, node distance=4cm] (latent) {Latent Representation};
    \node[trapezoid, shape border rotate=90, right of=latent, node distance=4cm] (decoder) {Decoder};
    \node[rect, right of=decoder, node distance=3cm] (output) {Output};

    \draw[arrow] (data) -- (encoder);
    \draw[arrow] (encoder) -- (latent);
    \draw[arrow] (latent) -- (decoder);
    \draw[arrow] (decoder) -- (output);

    \end{tikzpicture}
\end{figure}

\begin{multline*}
   \ML(\theta, \varphi) =
    \sum_{i=1}^{n} \int q(z_i \mid x_i, \varphi)
    \log \frac{p(x_i, z_i \mid \theta)}{q(z_i \mid x_i, \varphi)} \dd z_i =\\
    \sum_{i=1}^{n} \left[ \int q(z_i \mid x_i, \varphi) \log p(x_i \mid z_i, \theta) \dd z_i +
    \underbrace{\int q(z_i \mid x_i, \varphi) \log \frac{p(z)}{q(z_i \mid x_i, \varphi)} \dd z_i}_{-\operatorname{KL}(q(z_i \mid x_i, \varphi) || p(z_i))} \right]
\end{multline*}

\begin{equation*}
    \begin{aligned}
        \pdv{\theta}\ML(\theta, \varphi) =
        &\pdv{\theta} \sum_{i=1}^{n} \left[ \int q(z_i \mid x_i, \varphi) \log p(x_i \mid z_i, \theta) \dd z_i \right]
        \approx \\
        n &\pdv{\theta} \log p(x_j \mid \hat{z}, \theta)
        \text{ where } j \sim U \{1, \ldots, n\} \text{ and } z \sim q(z_j \mid x_j, \varphi) \\ 
        \\
        \pdv{\varphi}\ML(\theta, \varphi) =
        &\pdv{\theta} \sum_{i=1}^{n} \left[
            \int q(z_i \mid x_i, \varphi) \log p(x_i \mid z_i, \theta) \dd z_i
            -\operatorname{KL}(q(z_i \mid x_i, \varphi) || p(z_i))
        \right] \approx \\ 
        n &\pdv{\varphi} \log p(x_j \mid g(\widehat{\varepsilon}, \varphi), \theta) - n \pdv{\varphi} \operatorname{KL}( \ldots ) \text{ where } \widehat{\varepsilon} \sim \MN(0, I)
    \end{aligned}
\end{equation*}

$q$ depends on $\varphi$, so we can't differentiate directly, need \texttt{REINFORCE} or reparametrisation trick.
We take $z_i = g(\varepsilon, x_i, \varphi)$ where $\varepsilon \sim r(\varepsilon)$.
    \begin{multline*}
    ? = \pdv{\varphi} \sum_{i=1}^{n} \left[
        \int r(\varepsilon) \log p(x_i \mid g(\varepsilon, x_i, \varphi), \theta) \dd z_i
        -\operatorname{KL}(q(z_i \mid x_i, \varphi) || p(z_i))
    \right] =\\ n \pdv{\varphi} \log p(x_j \mid g(\varepsilon, x_j, \varphi), \theta) 
    - n \pdv{\varphi} \operatorname{KL}(q(z_j \mid x_j, \varphi) || p(z_j)) = ? 
\end{multline*}

Now that we have stochastic gradients, we can train our VAE, both the encoder $q(z \mid x, \varphi)$ and the decoder $p(x \mid z, \theta)$. \\ 

The first term in the expression for $\ML(\theta, \varphi)$, aka ELBO, corresponds to reconstruction error.
We thus require low reconstruction error on every training sample and prevent mode collapse problem, which GANs are prone to. \\ 

The second term (KL divergence) requires our approximate posterior $q$ to be not too far from the prior $p(z)$.
Without it, $q$ would converge to a delta function corresponding to MLE value for $z$ that maximizes $p(x\mid z, \theta)$, so this term mandates that the $q$ gaussians are not too narrow.
Also, without it inference (generation) and training would be too different, because during training we sample $z$s from $q$, but during training we don't use the encoder and only use the decoder, which receives $z$s from the prior $z$. 
In order to make this work, $q$ must not be not too far from the prior. \\ 

New idea. Take $x = f(z, \theta)$ to be a neural network which is invertible for each value of its parameters $\theta$.
Take a training sample:
\[
    x \in \RR^D, z \in \RR^D, X = (x_1, x_2, \ldots, x_n)
.\]
Consider the likelihood:
\[
    \log p(X \mid \theta) = \sum_{i=1}^{n} \log p(x_i \mid \theta) = 
    \sum_{i=1}^{n} \left[ p_z(f^{-1}(x_i, \theta)) + \log\abs{\det \dv{f^{-1}(x_i, \theta)}{x_i}} \right]
.\]
It takes $O(D^3)$ to calculate the determinant, too inefficient.
To remedy this, assume $x=f(z, \theta) = f_{\tau}(f_{\tau-1}(\ldots, f_1(z, \theta), \theta), \theta)$ where $f_{\tau}$ are relatively simple.
Then the $\log\det$ term decomposes into a sum of cheap-to-compute terms:
\[
    \log\abs{\det \dv{f^{-1}(x_i, \theta)}{x_i}} = \sum_{\tau=1}^{T} \ldots
\]
Two possible options to choose the form of $f_{\tau}$
\begin{itemize}
    \item simple transformations aka Normalizing flow (jacobian takes $O(D)$ to calculate)
    \item slight transformations aka NeuroODE (also kinda $O(D)$)
\end{itemize}

\subsection{Seminar}

Discussed formulas from \href{https://casmls.github.io/general/2017/04/24/iwae-aae.html}{this paper} on IWAE.
Then turned to specific definition of Normalizing Flows, i.e. how to choose the $f_{\tau}$s advertised during the lecture.
A \href{https://arxiv.org/pdf/1908.09257.pdf}{good tutorial} on Normalizing flows.
